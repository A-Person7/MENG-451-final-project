\documentclass{article} 
\special{papersize=8.5in,11in}

% Libraries
% \usepackage[margin=0in]{geometry}
\usepackage{geometry}

\usepackage{amsmath}
\usepackage{mathtools}
\usepackage{amssymb}
\usepackage{xfrac}
\usepackage{tabto}
\usepackage{tikz}

\usepackage[thinc]{esdiff}
\usepackage{float}
\usepackage{graphicx}
% pdfusetitle part means PDF title (metadata) defaults to the given title of the document
\usepackage[
    pdfusetitle,
    colorlinks=true,
    % linkcolor=blue
]{hyperref}
\usepackage{esdiff}
\usepackage{physics}

\usepackage{textcomp}
\usepackage{xcolor}
\usepackage{hyperref}
\usepackage[hyper]{listings}
\usepackage{booktabs}

\usepackage{amsmath}


\definecolor{codegray}{rgb}{0.5,0.5,0.5}
\definecolor{codepurple}{rgb}{0.58,0,0.82}
\definecolor{backcolour}{rgb}{0.97,0.97,0.97}
\definecolor{commentcolor}{rgb}{0.4,0.4,0.4}


\lstdefinestyle{generalcodestyle}{
    backgroundcolor=\color{backcolour},   
    commentstyle=\color{commentcolor},
    keywordstyle=\color{magenta},
    numberstyle=\tiny\color{codegray},
    stringstyle=\color{codepurple},
    basicstyle=\ttfamily\footnotesize,
    breakatwhitespace=false,         
    breaklines=true,                 
    captionpos=b,                    
    keepspaces=true,                 
    numbers=left,                    
    numbersep=5pt,                  
    showspaces=false,                
    showstringspaces=false,
    showtabs=false,                  
    tabsize=2
}
\lstset{
    language=[3]Python,
    style=generalcodestyle,
    showstringspaces=false,
    upquote=true,
    basicstyle=\fontencoding{T1}\selectfont,
    breaklines=true,
    breakatwhitespace=true,
    numbers=left,
    indexstyle=[1]\indexkeywords,
    % Experimental feature per the listings doc, doesn't seem to work as I expect anyways
    hyperref=[1]\indexkeywords,
}

% Must be loaded after hyperref
\usepackage{enumitem,cleveref}% http://ctan.org/pkg/{enumitem,cleveref}


\title{MENG 451 Final Project}
\author{Lucas Johnston}
\date{\today}

% gets rid of section numbering, but preserves hyperref indexes
\setcounter{secnumdepth}{0}

% No clue how the number here is interpeted. 1 does nothing, 2 is a giant space 
% https://tex.stackexchange.com/questions/198432/using-the-tab-command
\NumTabs{17}

\begin{document} 
    \maketitle

    \section{Introduction}

    % \tab~Source code is available in the \hyperref[sec:code]{Code} section using \href{https://www.python.org/}{Python 3} unless otherwise specified, and all original files, including the \LaTeX~source for this document, are available upon request. This document may include a brief, high-level description or analysis about each problem, with (most of) the implementation in \hyperref[sec:py]{hw8.py}.
    Hello world?
    \section{Problem Description}
    Consider a pendulum.\newline\newline
    While \href{https://canvas.gonzaga.edu/courses/24197}{fascinating} on its own, it can be made considerably more difficult by inverting.

    \begin{figure}[H]
        \centering
        \begin{tikzpicture}
            % Rectangle
            \coordinate (M) at (0,0);
            \coordinate (O) at (1.5,-1);
            \coordinate (G) at (1.2,1.2);
            \draw (M) -- (O)node[below]{O};
            \draw (O) -- (G)node[right]{G};
            % Center of Gravity
            \filldraw[fill=white](G) circle(0.1);
            \draw [fill=black] (G) -- ++(0.1,0) arc(0:90:0.1) -- cycle;
            \draw [fill=black] (G) -- ++(-0.1,0) arc(180:270:0.1) -- cycle;
            % \node [right = 2.0pt](M) {M};
            % \node [right = 2.0pt](0.5,-1) {O};
            % \node [right = 2.0pt](G) {G};


            % % Forces
            % \draw [-Stealth, violet] (0,-0.1 - \forceoffset) -- (0,-1.5) node[right] {$P_y$};
            % \draw [-Stealth, violet] (-0.1 - \forceoffset,0) -- (-1.75,0) node[left] {$P_x$};
            % \draw [-Stealth, violet] (0,-1.5-\forceoffset) -- (0,-2) node[below] {$m_{\textrm{p}} g$};
            % \draw [Stealth-Stealth, violet] (1.5+\forceoffset,0) -- (2.5,0) node[right] {$N$};

            % coordinate system
            \coordinate (a) at (4,0);
            \coordinate (b) at (5,0);
            \coordinate (c) at (4,1);
            \draw (a) -- (b)node[right]{$\hat{\imath}$};
            \draw (a) -- (c)node[above]{$\hat{\jmath}$};

        \end{tikzpicture}
        \caption{Free Body Diagram of the Piston}
    \end{figure}

    \section{Equations of Motion}
    Using Lagrangian mechanics, the equation of motion with respect to the variable of interest ($\phi$) can be expressed.
    % \begin{equation}
    %     \ddot{\vec{r}}_{\mathrm{CG}}
    %     =
    % \begin{bmatrix}- l_{\mathrm{CG}} \sin{\left(\phi \right)} \left(\dot \phi\right)^{2} + l_{\mathrm{CG}} \cos{\left(\phi \right)} \ddot \phi - r \sin{\left(\theta \right)} \left(\dot \theta\right)^{2} + r \cos{\left(\theta \right)} \ddot \theta\\- l_{\mathrm{CG}} \sin{\left(\phi \right)} \ddot \phi - l_{\mathrm{CG}} \cos{\left(\phi \right)} \left(\dot \phi\right)^{2} + r \sin{\left(\theta \right)} \ddot \theta + r \cos{\left(\theta \right)} \left(\dot \theta\right)^{2}\end{bmatrix}
    % \end{equation}
    \begin{equation}\label{eq:eom_raw}
       %  -  I \ddot \phi -  g l_{\mathrm{CG}} m \cos{\left(\phi \right)} -  l_{\mathrm{CG}}^{2} m \ddot \phi +  l_{\mathrm{CG}} m r \sin{\left(\phi + \theta \right)} \left(\dot \theta\right)^{2} -  l_{\mathrm{CG}} m r \cos{\left(\phi + \theta \right)} \ddot \theta
        I \ddot \phi +  l_{\mathrm{CG}} mg \cos{\left(\phi \right)} +  l_{\mathrm{CG}}^{2} m \ddot \phi -  l_{\mathrm{CG}} m r \sin{\left(\phi + \theta \right)} {\left(\dot \theta\right)}^{2} +  l_{\mathrm{CG}} m r \cos{\left(\phi + \theta \right)} \ddot \theta
        =
        0
    \end{equation}
    By inspecting Eq.~\eqref{eq:eom_raw}, the equilibrium solutions to the system can be found when $\cos{\left(\phi\right)} = 0$.
    For the unstable system posed, the task is now to develop a control system that minimizes the error between $\phi$ and $0$ and seeks the unstable equilibrium.

    Of course, Eq.~\eqref{eq:eom_raw} is not in a convenient form to simulate.
    Let
    \begin{equation}
        \begin{bmatrix}
            s_0 \\
            s_1
        \end{bmatrix}
        \equiv
        \begin{bmatrix}
            \phi
            \vspace{2pt}
            \\
            \dot{\phi}
        \end{bmatrix}
        \implies
        \diff{}{t}
        \begin{bmatrix}
            s_0 \\
            s_1
        \end{bmatrix}
        =
        \begin{bmatrix}
            s_1
            \vspace{2pt}
            \\
            \ddot{\phi}
        \end{bmatrix}
    \end{equation}

    % \newpage
    % \section{Code}\label{sec:code}
    % \subsection{hw8.py}\label{sec:py}
    % \begin{small}
    % \lstinputlisting[]{hw8.py}
    % \end{small}

\end{document}
