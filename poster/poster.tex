\documentclass[final]{beamer} 
\special{papersize=36.0in,48.0in}
% From https://github.com/academic-templates/tex-poster-template/
\usepackage{styles/poster}

% Use 
% $  latexmk -pvc -xelatex poster.tex 
% to compile this document.

% Libraries
% \usepackage[margin=0in]{geometry}
\usepackage{geometry}
\usepackage{adjustbox}

\usepackage{amsmath}
\usepackage{mathtools}
\usepackage{amssymb}
\usepackage{xfrac}
\usepackage{tabto}
\usepackage{tikz}

\usepackage[thinc]{esdiff}
\usepackage{float}
\usepackage{graphicx}
% pdfusetitle part means PDF title (metadata) defaults to the given title of the document
% \usepackage[
%     pdfusetitle,
%     colorlinks=true,
%     % linkcolor=blue
% ]{hyperref}
% \usepackage{physics}

\usepackage{textcomp}
\usepackage{xcolor}
\usepackage[hyper]{listings}
\usepackage{booktabs}

\usepackage{amsmath}


\definecolor{codegray}{rgb}{0.5,0.5,0.5}
\definecolor{codepurple}{rgb}{0.58,0,0.82}
\definecolor{backcolour}{rgb}{0.97,0.97,0.97}
\definecolor{commentcolor}{rgb}{0.4,0.4,0.4}


\lstdefinestyle{generalcodestyle}{
    backgroundcolor=\color{backcolour},   
    commentstyle=\color{commentcolor},
    keywordstyle=\color{magenta},
    numberstyle=\tiny\color{codegray},
    stringstyle=\color{codepurple},
    basicstyle=\ttfamily\footnotesize,
    breakatwhitespace=false,         
    breaklines=true,                 
    captionpos=b,                    
    keepspaces=true,                 
    numbers=left,                    
    numbersep=5pt,                  
    showspaces=false,                
    showstringspaces=false,
    showtabs=false,                  
    tabsize=2
}
\lstset{
    language=[3]Python,
    style=generalcodestyle,
    showstringspaces=false,
    upquote=true,
    basicstyle=\fontencoding{T1}\selectfont,
    breaklines=true,
    breakatwhitespace=true,
    numbers=left,
    indexstyle=[1]\indexkeywords,
    % Experimental feature per the listings doc, doesn't seem to work as I expect anyways
    hyperref=[1]\indexkeywords,
}

% Must be loaded after hyperref
\usepackage{enumitem,cleveref}% http://ctan.org/pkg/{enumitem,cleveref}


\title{MENG 451 Final Project}
\author{Lucas Johnston}
\date{\today}

% gets rid of section numbering, but preserves hyperref indexes
\setcounter{secnumdepth}{0}

% No clue how the number here is interpeted. 1 does nothing, 2 is a giant space 
% https://tex.stackexchange.com/questions/198432/using-the-tab-command
\NumTabs{17}

\begin{document} 
\begin{poster}
    \begin{column}{\onecolwidth} % The first column

    \begin{block}{Problem}
    % \tab~Source code is available in the \hyperref[sec:code]{Code} section using \href{https://www.python.org/}{Python 3} unless otherwise specified, and all original files, including the \LaTeX~source for this document, are available upon request. This document may include a brief, high-level description or analysis about each problem, with (most of) the implementation in \hyperref[sec:py]{hw8.py}.
    Consider a pendulum.\newline\newline
    While \href{https://canvas.gonzaga.edu/courses/24197}{fascinating} on its own, it can be made considerably more difficult by inverting.

    \begin{figure}[H]
        \centering
        \begin{tikzpicture}
            % Rectangle
            \coordinate (O) at (0,0);
            \coordinate (C) at (1.5*3,-1*3);
            \coordinate (G) at (1.2*3,1.2*3);
            \draw [very thick](O)node[left]{O} -- (C)node[below]{C};
            \draw [very thick](C) -- (G)node[right]{G};
            % Center of Gravity
            \filldraw[fill=white](G) circle(0.1);
            \draw [fill=black] (G) -- ++(0.1,0) arc(0:90:0.1) -- cycle;
            \draw [fill=black] (G) -- ++(-0.1,0) arc(180:270:0.1) -- cycle;
            % \node [right = 2.0pt](M) {M};
            % \node [right = 2.0pt](0.5,-1) {O};
            % \node [right = 2.0pt](G) {G};


            % % Forces
            % \draw [-Stealth, violet] (0,-0.1 - \forceoffset) -- (0,-1.5) node[right] {$P_y$};
            % \draw [-Stealth, violet] (-0.1 - \forceoffset,0) -- (-1.75,0) node[left] {$P_x$};
            % \draw [-Stealth, violet] (0,-1.5-\forceoffset) -- (0,-2) node[below] {$m_{\textrm{p}} g$};
            % \draw [Stealth-Stealth, violet] (1.5+\forceoffset,0) -- (2.5,0) node[right] {$N$};

            % coordinate system
            \coordinate (a) at (4+4,0);
            \coordinate (b) at (5+4,0);
            \coordinate (c) at (4+4,1);
            \draw [very thick](a) -- (b)node[right]{$\hat{\imath}$};
            \draw [very thick](a) -- (c)node[above]{$\hat{\jmath}$};

        \end{tikzpicture}
        \caption{Diagram of the (inverted) pendulum}
    \end{figure}
    \end{block}

    Here, O is a fixed point where a driven motor rotates C around. GC is a pendulum, and can be modeled with a bearing connection to OC\@.
    Let the angle $\theta$ describe OC's rotation with respect to a downwards vertical orientation (origin), and $\phi$ represent the angle between GC and upwards vertical.

    For an end goal of a physical system capable of stabilizing this (which is left to the reader), a model of the dynamics of such a system shall be generated and discussed, along with relevant selection processes for physical components.

    \section{Equations of Motion}
    Using Lagrangian mechanics, the equation of motion with respect to the variable of interest ($\phi$) can be expressed. Note that the equation generated by setting $q_k = \theta$ isn't included due to $\theta$ being determined by a motor speed controller, and modeled differently.
    \begin{multline}\label{eq:eom_raw}
        -  \mathrm{I} \ddot \phi +  g l_{CG} m \sin{\left(\phi \right)} -  l_{CG}^{\textrm{ }2} m \ddot \phi \\ +  l_{CG} m r \sin{\left(\phi + \theta \right)} \left(\dot \theta\right)^{2} -  l_{CG} m r \cos{\left(\phi + \theta \right)} \ddot \theta
        =
        0
    \end{multline}
    By inspecting Eq.~\eqref{eq:eom_raw}, the equilibrium solutions to the system can be found when $\sin{\left(\phi\right)} = 0$.
    For the unstable system posed, the task is now to develop a control system that minimizes the error between $\phi$ and $0$ and seeks the unstable equilibrium.

    Of course, Eq.~\eqref{eq:eom_raw} is not in a convenient form to simulate.
    Let
    \begin{equation}
        \begin{bmatrix}
            s_0 \\
            s_1
        \end{bmatrix}
        \equiv
        \begin{bmatrix}
            \phi
            \vspace{2pt}
            \\
            \dot{\phi}
        \end{bmatrix}
        % \implies
        % \diff{}{t}
        % \begin{bmatrix}
        %     s_0 \\
        %     s_1
        % \end{bmatrix}
        % =
        % \begin{bmatrix}
        %     s_1
        %     \vspace{2pt}
        %     \\
        %     \ddot{\phi}
        % \end{bmatrix}
    \end{equation}
    For a simple pendulum (all the mass is concentrated at the CG),
    \begin{equation}
        \omega_n \approx 
    \end{equation}
    \end{column}
    \begin{column}{\onecolwidth}
    \begin{block}{System parameters}
        \begin{table}[H]
        \caption[Table]{Parameters}\label{tab:params}
        \vspace{5pt}

        \centering{
            \begin{adjustbox}{width=1\textwidth}
            \small
            \begin{tabular}{ c  c  c}
            \midrule
                Parameter & Description & Test Value \\
            \midrule
                $m$ & Pendulum mass & $25\cdot10^{-3}$ kg\\
                $l$ & Pendulum total length & $50\cdot10^{-3}$ m\\
                $l_\textrm{CG}$ & C to Center of Gravity& $\sfrac{l}{2}$ \\
                $r$ & Length from O to C& $12.5\cdot10^{-3}$ m\\
                $\textrm{\upshape I}$ & Pendulum moment of inertia& $\frac{1}{3}ml^{\textrm{ }2}$\\
                $g$ & Gravitational acceleration& $9.80665$ ms$^{-2}$\\
            \hline
            \end{tabular}
            \end{adjustbox}
        }
    \end{table}

        Tweaking these parameters can be used to dial in system behavior and create a more or less difficult problem to stabilize. For example, by moving the center of mass for the pendulum closer to point C, it will rotate less quickly when out of equilibrium.
      \end{block}
      \begin{block}{Control Schemes}
        A variety of techniques are possible to use for such a system. Note that, due to the choice of origins, $\phi$ and $\theta$ are both 0 at the desired unstable equilibrium, and can be used directly to discuss the errors of a system.

        For a simple pendulum (all mass is coincident with the CG), the natural frequency for small rotations can be expressed as follows:
        \begin{equation}
            \omega_n \approx \sqrt{\frac{g}{l}}
        \end{equation}
          Notable techniques involve using a second order control system to achieve desired transient time, using a PD controller, and using rapid oscillation (a discussion of a form of the later approach can be found in \textit{The Simulation and Analysis of a Single and Double Inverted Pendulum with a Vertically-Driven Pivot} by Gustavo Lee). These control systems typically depend on the natural frequency of pendulum to determine required forcing frequencies. See the source code for more details.
      \end{block}
      \begin{block}{Simulation}
          A Python script is provided to simulate the motion of the pendulum system in response to changing system parameters or control schemes for $\theta$. A typical Runge-Kutta 4(5) integration scheme was chosen due to its general accuracy and speed.
          Dry and viscous friction effects were omitted from the simulation as they were not expected to alter system dynamics strongly when the system was at its unstable equilibrium.
      \end{block}
    \end{column}
    \separator
    \begin{column}{\onecolwidth}
        \begin{figure}[H]
            \includegraphics[trim={1.5in, 0,0,0}, clip, scale=0.8]{../figs/example_simulation.png}
            \caption{Snapshot of simulated animation}
        \end{figure}

  \begin{block}{Physical Implementation}
      Once system parameters discussed in Table~\ref{tab:params} have been decided for an application, and the simulation has a desired (transient) response, motor selection can be made based on simulated torque requirements and the max speed at which the theoretical control system can produce. Feedback on angular position of the pendulum can be determined through a potentiometer with hard stops to prevent damage (which would limit the range of motion of the pendulum), a rotary encoder, or optical sensors. Acceptable time delays for these sensors depend on the frequency of the system selected. 
        \begin{figure}[H]
            \includegraphics{../figs/drawing_tree.drawio}
            \caption{Suggested engineering drawing tree for full implementation}
        \end{figure}
  \end{block}
  \begin{block}{Real World Implications}
    Inverted pendulum balancing has important real-world applications for several dynamically unstable systems, including rocketry and robots. A stable rocket launch requires maintaining a vertical trajctory and setting up system dynamics and mass balance to achieve that. Robots, especially humanoid bipedal robots, require adjustments to remain upright during gait planning and execution.
  \end{block}
    \begin{block}{Code and Document Sources}
    \begin{figure}[H]
        \includegraphics{../figs/qr_code_to_repo.png}
        \caption{Link to git repository}
    \end{figure}
  \end{block}
\end{column}

    % \newpage
    % \section{Code}\label{sec:code}
    % \subsection{hw8.py}\label{sec:py}
    % \begin{small}
    % \lstinputlisting[]{hw8.py}
    % \end{small}

\end{poster}
\end{document}
